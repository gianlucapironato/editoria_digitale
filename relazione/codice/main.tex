\documentclass[a4paper,12pt]{article}
\usepackage[utf8]{inputenc}
\usepackage{graphicx}
\usepackage{hyperref}
\usepackage{fancyhdr}
\usepackage{amsmath}
\usepackage{amsfonts}
\usepackage{amssymb}

\title{Relazione del progetto di Editoria Digitale \\ "Orizzonti spezzati: racconti di fantascienza"}
\author{Gianluca Pironato \\ 47257A}
\date{10 settembre 2024}

\begin{document}

\clearpage\maketitle
\thispagestyle{empty}

\begin{figure}[h]
    \centering
    \includegraphics[width=0.3\textwidth]{unimi.pdf} % Inserire il logo corretto
    \caption{Logo UNIMI}
\end{figure}

\newpage

\setcounter{page}{1}

\section*{Introduzione}

Il progetto ruota attorno allo sviluppo di una raccolta di racconti di fantascienza generati dall'intelligenza artificiale mediante l'impiego di modelli per la creazione di testi come ChatGPT e strumenti per la conversione tra formati come Pandoc. Il risultato è un prototipo contenente tre racconti appartenenti al genere della fantascienza, su un totale di dieci ipotizzati nella raccolta completa, e opportunamente formattati per la distribuzione digitale. Il progetto evidenzia l'efficacia delle applicazioni AI alle fasi editoriali di acquisizione dei contenuti e di produzione dei formati di output.

\section*{Ideazione}
\subsection*{Tema}

Si è deciso di sviluppare una raccolta di racconti, piuttosto che un singolo romanzo o un'altra tipologia di prodotto editoriale, principalmente per la flessibilità che questo formato offre. Una raccolta di racconti consente infatti di esplorare una varietà di temi, stili e ambientazioni, permettendo una maggiore sperimentazione narrativa e stilistica. Inoltre, il formato dei racconti brevi riduce il rischio complessivo del progetto: ogni racconto è una storia indipendente, e un eventuale fallimento di uno non compromette l'intera opera, sia dal punto di vista degli errori eventuali commessi dal modello utilizzato per la generazione dei testi, sia dal punto di vista del possibile gradimento da parte dei lettori finali.

\bigbreak

Per quanto riguarda la scelta del genere, la fantascienza è stata individuata come il contesto ideale per il progetto grazie alla sua capacità di catturare l'immaginazione dei lettori con scenari futuristici, tecnologie avanzate e mondi alternativi. I racconti sono però legati da altri temi, quali le le relazioni interpersonali, la solitudine e l'identità.

\bigbreak

La decisione di utilizzare ambientazioni fantascientifiche per esplorare temi umani e universali, seppur in contesti straordinari, consente di mantenere il genere fantascientifico come elemento di attrazione per i lettori, offrendo allo stesso tempo storie che toccano corde emotive profonde, rendendo il prodotto finale non solo affascinante dal punto di vista narrativo, ma anche ricco di significati e interpretazioni.


\subsection*{Destinatari}


Come già si è evidenziato, la raccolta è stata progettata per attrarre un pubblico eterogeneo, che spazia dagli appassionati del genere ai lettori più occasionali, offrendo storie che non solo intrattengono ma stimolano anche riflessioni più profonde. Le personas identificate rappresentano profili distinti di potenziali lettori, ciascuno con motivazioni e interessi specifici, e gli scenari d'uso mostrano come questa raccolta possa essere apprezzata pressoché ovunque, grazie ai dispositivi di lettura.

Ecco dunque una lista di possibili destinatari:

\begin{itemize}
    \item Luca, 34 anni, ingegnere informatico, appassionato di tecnologia e fantascienza. Apprezza le storie che esplorano il futuro della scienza e della tecnologia, con un particolare interesse per i racconti che lo fanno riflettere sulle implicazioni etiche e filosofiche delle innovazioni;
    \item Maria, 28 anni, insegnante di letteratura, ama leggere, anche se la fantascienza non è il suo genere preferito. È affascinata da storie che esplorano le emozioni umane;
    \item Giovanni, 42 anni, manager, lettore occasionale che preferisce storie brevi da leggere nei momenti di pausa. Non è particolarmente interessato alla fantascienza, ma è curioso di esplorare come questo genere possa offrire.
\end{itemize}

Luca legge uno dei racconti durante la pausa pranzo al lavoro, trovando nella fantascienza un modo per evadere dalla routine e riflettere su innovazioni tecnologiche e dilemmi etici. Maria, in un pomeriggio di relax nel weekend, scopre che un racconto, pur in un contesto futuristico, risuona con le sue esperienze personali, offrendo introspezione e intrattenimento. Giovanni, durante un viaggio in treno, apprezza la brevità e l'intensità delle storie, che gli permettono di rilassarsi e riflettere su aspetti quotidiani da una nuova prospettiva. Questi scenari mostrano come la raccolta si adatti a diversi momenti della vita dei lettori, rispondendo alle loro esigenze di intrattenimento e riflessione.

\section*{Requisiti di accettazione}
Indicate i requisiti di accettazione che dovranno essere soddisfatti per raggiungere i destinatari. 

\section*{Canali di distribuzione}
Presentare i canali di distribuzione che si intendono raggiungere e i formati dati richiesti da ogni canale. 

\section*{Processo di Produzione}
\subsection*{Acquisizione dei contenuti}
Descrivere le fonti che saranno utilizzate nella costruzione del prodotto editoriale.

\subsection*{Gestione documentale}
Descrivere il flusso di gestione documentale definito per il progetto.

\subsection*{Tecnologie adottate}
Descrivere le tecnologie addottate nelle diverse fasi e discuterne il contributo.

\section*{Valutazione dei risultati raggiunti}
\subsection*{Valutazione del flusso di produzione}
Per valutare il contributo proposto valutare le diverse fasi del flusso.

\subsection*{Confronto con lo stato dell’arte}
Confrontare una versione AS-IS del flusso di gestione e una TO-BE.

\subsection*{Limiti emersi}
Sottolineare i limiti emersi durante il progetto.

\section*{Conclusioni}
Discutere i risultati ottenuti, verificando se gli obiettivi definiti siano stati raggiunti.

\end{document}
