\documentclass[a4paper,12pt]{article}
\usepackage[utf8]{inputenc}
\usepackage{graphicx}
\usepackage{hyperref}
\usepackage{fancyhdr}
\usepackage{amsmath}
\usepackage{amsfonts}
\usepackage{amssymb}
\usepackage{listings}

\title{Relazione del progetto di Editoria Digitale \\ "Orizzonti spezzati: racconti di fantascienza"}
\author{Gianluca Pironato \\ 47257A}
\date{10 settembre 2024}

\begin{document}

\clearpage\maketitle
\thispagestyle{empty}

\begin{figure}[h]
    \centering
    \includegraphics[width=0.3\textwidth]{unimi.pdf} % Inserire il logo corretto
    \caption{Logo UNIMI}
\end{figure}

\newpage

\setcounter{page}{1}

\section*{Introduzione}

Il progetto ruota attorno allo sviluppo di una raccolta di racconti di fantascienza generati dall'intelligenza artificiale mediante l'impiego di modelli per la creazione di testi come ChatGPT e strumenti per la conversione tra formati come Pandoc. Il risultato è un prototipo contenente tre racconti appartenenti al genere della fantascienza, su un totale di dieci ipotizzati nella raccolta completa, e opportunamente formattati per la distribuzione digitale. Il progetto evidenzia l'efficacia delle applicazioni di IA alle fasi editoriali di acquisizione dei contenuti e di produzione dei formati di output.

\section*{Ideazione}
\subsection*{Tema}

Si è deciso di sviluppare una raccolta di racconti, piuttosto che un singolo romanzo o un'altra tipologia di prodotto editoriale, principalmente per la flessibilità che questa offre. Una raccolta di racconti consente, infatti, di esplorare una varietà di temi, stili e ambientazioni, permettendo una maggiore sperimentazione narrativa e stilistica. Inoltre, il formato dei racconti brevi riduce il rischio complessivo del progetto: ogni racconto è una storia indipendente, e l'eventuale fallimento di uno non compromette l'intera opera, sia dal punto di vista degli errori potenzialmente commessi dal modello utilizzato per la generazione dei testi, sia dal punto di vista del possibile gradimento da parte dei lettori finali.

\bigbreak

Per quanto riguarda la scelta del genere, la fantascienza è stata individuata come il contesto ideale per il progetto grazie alla sua capacità di catturare l'immaginazione dei lettori con scenari futuristici, tecnologie avanzate e mondi alternativi. I racconti sono però legati da altri temi, quali le le relazioni interpersonali, la solitudine e l'identità.

\bigbreak

La decisione di esplorare temi umani e universali, seppur in contesti straordinari, consente di mantenere il genere fantascientifico come elemento di attrazione per i lettori, offrendo allo stesso tempo storie che toccano corde emotive profonde, rendendo il prodotto finale non solo affascinante dal punto di vista narrativo, ma anche ricco di significati e interpretazioni.


\subsection*{Destinatari}


Come già si è evidenziato, la raccolta è stata progettata per attrarre un pubblico eterogeneo, che spazia dagli appassionati del genere ai lettori più occasionali, offrendo storie che non solo intrattengono ma stimolano anche riflessioni più profonde. Le personas identificate rappresentano profili distinti di potenziali lettori, ciascuno con motivazioni e interessi specifici, e gli scenari d'uso mostrano come questa raccolta possa essere apprezzata pressoché ovunque, grazie ai dispositivi di lettura.

Ecco dunque una lista di possibili destinatari:

\begin{itemize}
    \item Luca, 34 anni, ingegnere informatico, appassionato di tecnologia e fantascienza. Apprezza le storie che esplorano il futuro della scienza e della tecnologia, con un particolare interesse per i racconti che lo fanno riflettere sulle implicazioni etiche e filosofiche delle innovazioni;
    \item Maria, 28 anni, insegnante di letteratura, ama leggere, anche se la fantascienza non è il suo genere preferito. È affascinata da storie che esplorano le emozioni umane;
    \item Giovanni, 42 anni, manager, lettore occasionale che preferisce storie brevi da leggere nei momenti di pausa. Non è particolarmente appassionato alla fantascienza, ma è curioso di esplorare come questo genere possa offrire.
\end{itemize}

Luca legge uno dei racconti durante la pausa pranzo al lavoro, trovando nella fantascienza un modo per evadere dalla routine e riflettere su innovazioni tecnologiche e dilemmi etici. Maria, in un pomeriggio di relax nel weekend, scopre che un racconto, pur in un contesto futuristico, risuona con le sue esperienze personali, offrendo introspezione e intrattenimento. Giovanni, durante un viaggio in treno, apprezza la brevità e l'intensità delle storie, che gli permettono di rilassarsi e riflettere su aspetti quotidiani da una nuova prospettiva. Questi scenari mostrano come la raccolta si adatti a diversi momenti della vita dei lettori, rispondendo alle loro esigenze di intrattenimento e riflessione.

\subsection*{Requisiti di accettazione}

In linea con le aspettative e le esigenze dei destinatari individuati, si è stabilita innanzitutto una lunghezza dei racconti compresa tra 2000 e 5000 parole, così da assicurare lo spazio necessario allo sviluppo di una trama, senza però richiedere un impegno eccessivo in termini di tempo al lettore. Quest'ultimo infatti dev'essere in grado di optare per letture brevi e facilmente interrompibili. Per analoghi motivi, si è pensato di includere un numero totale di dieci racconti nella versione finale della raccolta. 

\bigbreak 

Dal punto di vista delle tecnologie, è immediato ipotizzare che l'esperienza di lettura avvenga mediante dispositivi eterogenei quali eReader, tablet o smartphone, per i quali il formato che garantisce più larga compatibilità è certamente ePub; inoltre, questo formato è conforme agli standard internazionali di accessibilità, rendendo i contenuti fruibili anche per i lettori con disabilità visive.

\bigbreak

La pubblicazione della raccolta può avvenire attraverso i principali canali digitali, per raggiungere il più vasto pubblico possibile. Le piattaforme di distribuzione più efficaci includono Amazon Kindle Store, Apple Books, Google Play Books e lo store Kobo. Queste piattaforme non solo offrono una vasta diffusione, ma anche strumenti che permettono ai lettori di scoprire nuovi contenuti in base ai loro interessi, aumentando così le possibilità che la raccolta venga consigliata a lettori di opere simili.

\section*{Processo di Produzione}
\subsection*{Acquisizione dei contenuti}

Il contenuto della raccolta, nel prototipo qui limitato al numero di tre racconti, è stato interamente creato e strutturato in capitoli mediante l'uso di ChatGPT, nella sua versione personalizzata chiamata Ebook Creator, riducendo così i tempi da dedicare alla fase che altrimenti risulterebbe la più onerosa nel processo di produzione editoriale. Il modello di linguaggio è stato opportunamente "istruito" circa i temi, lo stile e il genere di testo da generare ed è possibile consultare \href{https://github.com/gianlucapironato/editoria_digitale/blob/main/prompt%20acquisizione%20contenuti.md}{\underline{la lista di prompt adoperati nel repository allegato}}.

\bigbreak

Oltre a limitare la generazione al dominio fantascientifico, si è individuato anche l'insieme di temi trasversali ai racconti, con particolare riguardo alle relazioni interpersonali e alle dinamiche sociali. La generazione ha ovviamente attraversato diversi tentativi prima di giungere al prototipo finale: in primo luogo, si è ricercata una terna di racconti sufficientemente diversi, perciò alcune proposte troppo simili tra loro sono state escluse; in secondo luogo, si è intervenuti con prompt più guidati, atti a sottolineare l'importanza della varietà dei contenuti della raccolta, i cui motivi sono meglio spiegati più avanti, nella sezione dedicata ai limiti emersi.

\subsection*{Progettazione grafica}
Per quanto riguarda la veste grafica, nell'ottica di rendere più accattivante il prodotto si è pensato di affidarsi ad applicativi basati su IA per generare la copertina della raccolta e almeno un'immagine che rendesse l'atmosfera di sfondo di ciascun racconto. 

\bigbreak

Per la generazione di queste immagini è stato utilizzato lo strumento chiamato Runway, nella sua funzionalità text-to-image e, nello specifico, è stato creato un prompt diverso per ogni racconto, basato sulla corrispondente sinossi. \href{https://github.com/gianlucapironato/editoria_digitale/blob/main/immagini/prompt%20immagini.md}{\underline{I prompt sono consultabili nel file del repository allegato.}}

\subsection*{Trasformazione dei formati}
La trasformazione dei formati è stata necessaria per convertire i racconti in un formato idoneo alla distribuzione digitale. L'output testuale di ChatGPT è stato richiesto in formato Markdown, linguaggio di marcatura semplice e strutturato che consente di gestire velocemente titoli, sezioni e altri elementi di formattazione.

\bigbreak

Per trasformare il file Markdown contenente i racconti in un ebook in formato ePub, è stato utilizzato Pandoc, uno strumento versatile per la conversione di documenti. Il comando Pandoc utilizzato per la conversione è stato:

\begin{lstlisting}
pandoc --toc --metadata-file="metadati.txt" 
--epub-cover-image="immagini\copertina.jpg"
-o '.\orizzonti spezzati.epub' '.\orizzonti spezzati.md'
\end{lstlisting}

Il file denominato $metadati.txt$ contiene alcune informazioni in formato YAML sull'ebook quali titolo, autore, data di pubblicazione, descrizione e parole chiave utili per l'indicizzazione e per la distribuzione del prodotto finale; inoltre, le opzioni $toc$ e $epub-cover-image$ permettono di generare l'indice dei capitoli e di aggiungere un'immagine di copertina.


\subsection*{Esecuzione del flusso}
I prodotti del flusso di gestione documentale sono tutti consultabili attraverso il \href{https://github.com/gianlucapironato/editoria_digitale/}{\underline{repository del progetto.}}

\section*{Valutazione dei risultati raggiunti}

Gli obiettivi prefissati sono stati raggiunti perché la raccolta di racconti generati con IA comprende contenuti coerenti e in linea con le esigenze dei destinatari individuati. Dal punto di vista tecnico, l'adozione del formato ePub si è rivelata agevole anche nella fase di trasformazione automatizzata dei formati, perché il tutto si è ridotto all'esecuzione di un semplice comando Pandoc. 

\bigbreak

Nonostante i risultati positivi ottenuti, l'uso dell'IA per la generazione dei testi presenta alcuni limiti che devono essere attentamente considerati. Uno dei principali aspetti critici riguarda la necessità di una revisione accurata dei testi generati. In alcuni casi, ChatGPT può commettere errori ortografici, grammaticali o di coniugazione dei verbi, che richiedono un controllo manuale approfondito per garantire la qualità finale del prodotto.

\bigbreak

Un altro limite emerso durante la generazione dei racconti è la tendenza di ChatGPT a generare racconti che possono risultare simili a storie già apparse in altre opere letterarie. Questo fenomeno è legato all'addestramento svolto su vasti dataset di testi già esistenti, il che può portare alla ripetizione di schemi narrativi o tematiche già esplorate. Di conseguenza, c'è il rischio che i racconti generati manchino di originalità e innovazione.

\bigbreak

Inoltre, com'è visibile nello storico dei prompt utilizzati, è emerso anche un limite tematico sulle relazioni umane, spesso declinate prevalentemente dal punto di vista delle relazioni amorose di coppia. Questa focalizzazione può portare a una certa omogeneità tra i racconti, riducendo la varietà tematica e stilistica della raccolta. Per ovviare a questo limite, è stato necessario intervenire manualmente, con prompt ad hoc per diversificare le tematiche e ampliare la gamma di relazioni esplorate nei racconti, al fine di offrire un prodotto più variegato e interessante per i lettori.


\section*{Conclusioni}

In sintesi, sebbene l'uso dell'IA offra un guadagno di tempo assolutamente non trascurabile, è comunque da mettere in conto un attento processo di revisione e controllo per mitigare i suoi limiti. La supervisione umana è essenziale per correggere errori, garantire l'originalità dei testi e assicurare una varietà tematica che risponda alle aspettative del pubblico.

\end{document}
