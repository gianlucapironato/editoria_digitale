\documentclass[a4paper,12pt]{article}
\usepackage[utf8]{inputenc}
\usepackage{graphicx}
\usepackage{hyperref}
\usepackage{fancyhdr}
\usepackage{amsmath}
\usepackage{amsfonts}
\usepackage{amssymb}

\title{Relazione del progetto di Editoria Digitale \\ "Orizzonti spezzati: racconti di fantascienza"}
\author{Gianluca Pironato \\ 47257A}
\date{10 settembre 2024}

\begin{document}

\clearpage\maketitle
\thispagestyle{empty}

\begin{figure}[h]
    \centering
    \includegraphics[width=0.3\textwidth]{unimi.pdf} % Inserire il logo corretto
    \caption{Logo UNIMI}
\end{figure}

\newpage

\setcounter{page}{1}

\section*{Introduzione}
Breve descrizione del progetto toccando i punti più importanti affrontati nel documento. Obiettivi, tecnologie, aspetti salienti del flusso di gestione documentale, risultati raggiunti.

\section*{Ideazione}
\subsection*{Tema}
Identificazione dei temi che il prodotto editoriale dovrà presentare. Evidenziare gli argomenti correlati e la tendenza dell’attenzione su questi temi.

\subsection*{Destinatari}
Descrivere i destinatari del prodotto editoriale descrivendo le personas alle quali si rivolge il prodotto. Descrivete alcuni scenari d’uso nei quali inserire le personas scelte come destinatari.


\section*{Requisiti di accettazione}
Indicate i requisiti di accettazione che dovranno essere soddisfatti per raggiungere i destinatari. 

\section*{Canali di distribuzione}
Presentare i canali di distribuzione che si intendono raggiungere e i formati dati richiesti da ogni canale. 

\section*{Processo di Produzione}
\subsection*{Acquisizione dei contenuti}
Descrivere le fonti che saranno utilizzate nella costruzione del prodotto editoriale.

\subsection*{Gestione documentale}
Descrivere il flusso di gestione documentale definito per il progetto.

\subsection*{Tecnologie adottate}
Descrivere le tecnologie addottate nelle diverse fasi e discuterne il contributo.

\section*{Valutazione dei risultati raggiunti}
\subsection*{Valutazione del flusso di produzione}
Per valutare il contributo proposto valutare le diverse fasi del flusso.

\subsection*{Confronto con lo stato dell’arte}
Confrontare una versione AS-IS del flusso di gestione e una TO-BE.

\subsection*{Limiti emersi}
Sottolineare i limiti emersi durante il progetto.

\section*{Conclusioni}
Discutere i risultati ottenuti, verificando se gli obiettivi definiti siano stati raggiunti.

\section*{Bibliografia e sitografia}
\begin{itemize}
    \item L. Sechi, \textit{Editoria digitale}. Apogeo, 2010. \url{https://www.apogeonline.com/libri/editoria-digitale-letizia-sechi/}
    \item L. Pantieri, \textit{LaTeX per l’impaziente}. 2021. \url{http://www.lorenzopantieri.net/LaTeX_files/LaTeXimpaziente.pdf}
    \item P. Ceravolo, \textit{Lezioni di editoria digitale}. 2023. \url{https://myariel.unimi.it/mod/folder/view.php?id=26538}
\end{itemize}

\end{document}
